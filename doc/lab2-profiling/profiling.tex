\documentclass[a4paper,11pt]{article}
\usepackage{url}
\usepackage{hyperref}
\usepackage{listings}
\usepackage{color}
\usepackage{minted}
\usepackage{tikz}
\usepackage[T1]{fontenc}
\usetikzlibrary{arrows}
\usetikzlibrary{patterns}

\hoffset 0in
\oddsidemargin 0in
\voffset -0.4in
\topmargin 0in
\headheight 12pt
\headsep 0,5in

\marginparwidth 50pt
\marginparsep 5pt
\reversemarginpar


\textwidth 6.5in
\displaywidth 6.5in
\textheight 240mm
\parindent 0mm
\parskip \baselineskip

\newcommand{\code}[1]{\texttt{#1}}

%\newcommand{\heading}[1]{\vspace{2ex}\section*{#1}}
\newcommand{\refsection}[1]{Section \ref{#1}}


\begin{document}

\title{ \bf ENEL464 - Profiling }
\author{}
%\author{Michael Hayes}
\date{}
\maketitle


\section{Introduction}

When writing a program to solve a problem, you often want to make the program
run faster. The process of improving the runtime, or memory usage, or various
resource usage, is called \emph{optimising}. There are many ways to optimise
your code: rewriting complex math expressions; using clever data structures to
speed memory access; enabling built-in compiler optimisations; etc. However,
before any of that can occur, you must first identify \emph{where} to optimise.
For a program that takes two hours, there is no point optimising a chunk of code
that is responsible for 10 seconds of it (see Amdahl's law). You must first
\emph{profile} your code to identify where the CPU is spending most of its time
(so called \emph{hot-spots}) and \emph{why} it takes so much time.

\section{GProf}

GProf is a Unix profiling system. It uses a combination of
compile-time instrumentation (compiled into your program by GCC), and
run-time sampling (pauses your program and records the program
counter) to build reports on the execution details of a program.

There are three steps to using GProf:
\begin{enumerate}
    \item Compile your code with profiling enabled.
    \item Execute your program to collect runtime data.
    \item Process the execution report and analyse the code.
\end{enumerate}

\subsection{Compiling with profiling enabled}

From the man page of gcc:
%
\begin{minted}{shell}
    -pg : Generate extra code to write profile information suitable for the
    analysis program gprof. You must use this option when compiling the source
    files you want data about, and you must also use it when linking.
\end{minted}

So following this, we could modify the compilation command for the poisson
program to be:
%
\begin{minted}{shell}
$ g++ -pg -g -o poisson poisson.c -lpthread
\end{minted}
%
The \texttt{-g} option is required to include debugging symbols for annotating the program.


\subsection{Executing with profiling}

This is the same as executing the program normalling. Simply run
\begin{minted}{shell}
$ ./poisson -n 101 -i 100
\end{minted}

\emph{Note: when using statistical profilers, the more snapshots taken, the more
accurate the profiling will be.}

After executing the program, a new file called `gmon.out' should be present.
This contains the runtime data and will be used in the next step.

\subsection{Processing the runtime data}

The GProf tool can now be used to analyse the output from the previous step.

\begin{minted}{shell}
$ gprof poisson gmon.out [> file_to_dump_to.txt]
\end{minted}

This will list the most expensive functions in your program with total time
used, time spent in child calls, number of calls, etc... GProf can also use this
data to annotate the original source files:

\begin{minted}{shell}
$ gprof -A poisson gmon.out
\end{minted}

\emph{Note: this does not modify the original files but prints the annotated
version to stdout, just like the previous command.}

Now that you know where the program is spending its time, you need to figure out
\emph{why}? Is it a computer architectural reason? Perhaps you could write your
code differently to reduce the runtime? Can you organise your data differently?



\section{Cachegrind}

Cachegrind is a tool within the Valgrind suite. It works by simulating a modern
processor cache and recording all of the memory accesses performed by your
program. It will then list the various accesses to the level 1 data and
instruction caches as well as the lower level shared cache.

\emph{Note: because cachegrind \emph{simulates} your cache, it runs slowly so
do not run large computations with it.}

Cachegrind can be run using:

\begin{minted}{shell}
$ valgrind --tool=cachegrind ./poisson -n 51 -i 10
\end{minted}

This will generate and display some statistics about the cache usage. Usually,
for a computational problem, there will be very few instruction misses as the
code is short and run many times. However, there will be significant data misses
as your program traverses your simulation volume. Running cachegrind will also
generate a `cachegrind.out.<pid>' file, similar to that of the GProf output.
This can be used to annotate your source code using

\begin{minted}{shell}
$ cg_annotate cachegrind.out.<pid> [> cg_output.txt]
\end{minted}

The output from this command is quite wide, so storing it in a file is
recommended.


\section{Profiling with compiler optimisations}

Compiler optimisations work by stripping out excess instructions,
reordering, and simlifying the generated machine instructions. The
only rule the optmiser follows is that the generated code must have
the same output ``as if'' it had not been optimised at all. This gives
the compiler a lot of leeway and often results in very confusing
assembly outputs, see the note on optimisations for examples.

Because of this, certain functions may not exist in your compiled program
anymore, and instructions no longer correspond to specific source code lines.
This can be tricky when trying to profile as it becomes difficult to determine
where the hot-spot is located. However, profiling without optimisations does not
give you an accurate picture of real world program usage.

TLDR: you should profile your programs both with, and without, optimisation
as this will help give you a full picture of how your program runs.



\end{document}
