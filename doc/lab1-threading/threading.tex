\documentclass[a4paper,11pt]{article}
\usepackage{url}
\usepackage{hyperref}
\usepackage{listings}
\usepackage{color}
\usepackage{csquotes}
\usepackage{tikz}
\usepackage[T1]{fontenc}
\usetikzlibrary{arrows}
\usetikzlibrary{patterns}

\hoffset 0in
\oddsidemargin 0in
\voffset -0.4in
\topmargin 0in
\headheight 12pt
\headsep 0,5in

\marginparwidth 50pt
\marginparsep 5pt
\reversemarginpar


\textwidth 6.5in
\displaywidth 6.5in
\textheight 240mm
\parindent 0mm
\parskip \baselineskip

\newcommand{\code}[1]{\texttt{#1}}

%\newcommand{\heading}[1]{\vspace{2ex}\section*{#1}}
\newcommand{\refsection}[1]{Section \ref{#1}}


\begin{document}

\title{ \bf ENEL464 - Threading }
\author{}
%\author{Michael Hayes}
\date{}
\maketitle


\section{Introduction}

Often when writing computationally heavy applications, it is desirable to
leverage the multicore nature of modern CPUs to run calculations in parallel.
This can be done inside a single process by using \emph{threads}. A
\emph{thread} starts at a specified function and continues to execute until that
function returns. Your program will always have at least one, `main' thread
running the main() function. When this thread exits, all other threads are
terminated. Threads are useful because they execute in parallel. The operating
system can schedule threads to be run on different CPU cores simultaneously
allowing for up to $n$ times speed improvements (for $n$ cores) \footnotemark.

The operating system kernel provides functions to create new threads in your
program on the fly. In C on Linux, this is provided through the POSIX Threads
(pthreads) library, and in C++ using the standard library (\code{thread}). Both
of these methods take a pointer to a function that is executed as soon as the
thread is created. See \code{demos/threads.c} for an example using POSIX
threads. It works by creating a collection of worker thread which solve the
problem, and then waiting until they have finished. The waiting method is the
\code{join()} function which suspends the main thread and so does not waste
extra CPU time, and then returns once the thread has finished (joined).

\footnotetext{Note: threading (or multithreading) is different from Intel
HyperThreading or AMD Simultaneous Multithreading. These are both CPU hardware
architectures that allow for single physical core to appear as two logical cores
to the operating system. This will affect your programs, but in a different
way.}

\section{Next steps}

\begin{enumerate}
    \item Play with \code{threads.c} to understand how it works.
    \item Write a new program from scratch that uses multithreading,
      e.g., print out the numbers 1 through 10, each number in a
      different thread.
    \item Experiment with passing arguments to the worker threads.
    \item Consider how multithreading could be applied to the assignment.
\end{enumerate}

\section{Dangers}

Multithreading can be a very effective means of improving program performance,
however, there are some risks associated with this. Different sections of code
can be running simultaneously on different CPU cores in your computer. This
means that two threads could try to modify the same piece of memory at the same
time. Worse, thread behaviour can also depend on the order in which the threads
are running. This is called a race condition: two threads, A and B, are racing
to completion, but the behaviour changes if A wins or B wins. This can lead to
very complicated and hard to diagnose bugs.

Some of these risks can be mitigated using \emph{synchronisation primitives}.
These are kernel constructs that guarentee behaviour across threads. For
example, the \emph{mutex} can be used to make a resource
\emph{mutually-exclusive}, that is, to allow acces to only one thread at a time.
\emph{Semaphores} and \emph{condition variables} are further examples. In the
context of the 464 assignment, care must be taken to ensure that all
calculations are being done from the same iteration. This is because an
iteration depends on the previous state, thus they cannot be calculated at the
same time. This does not apply \emph{within} the iteration though.


\section{Troubleshooting}

When running into trouble with multithreading, it is often helpful to scale back
towards a minimum workable example until the problem disapears. For example, if
running into issues with 10 threads, does running with 1 child thread reproduce
the issue? Does running the same code in just the main thread work? Try to
narrow down where the problem is occuring. Perhaps you are trying to
simultaneously access a shared resource (remember: variables/memory are a shared
resource too!) and either need to ensure they are not being simultaneously read
and written to, or add a mutual exclusion barrier.

\section{Further reading}

\begin{itemize}
    \item \url{https://man7.org/linux/man-pages/man7/pthreads.7.html}. You can
    also view this man page in the command line by running \code{man pthreads}.
    Documentation for specific functions (e.g. \code{pthread\_create}) can be
    viewed by running \code{man <function-name>}.
    \item \url{http://www.cs.kent.edu/~ruttan/sysprog/lectures/multi-thread/multi-thread.html}
    \item \url{https://www.geeksforgeeks.org/multithreading-c-2/}
    \item \url{https://en.cppreference.com/w/cpp/thread/thread}
    \item \url{https://en.wikipedia.org/wiki/Thread_safety}
\end{itemize}


\end{document}
